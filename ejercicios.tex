\documentclass[11pt,spanish]{article} % Idioma
\usepackage{babel}
\usepackage[T1]{fontenc}
\usepackage{textcomp}
\usepackage[utf8]{inputenc} % Puede depender del instrucción, sistema o editor
\usepackage{wrapfig} % Imagenes
% \graphicspath{ {./imagenes/} }

\usepackage[left=2.75cm,top=2.5cm,right=2cm,bottom=2.5cm]{geometry} % Márgenes
%\usepackage{pstricks} % Gráficas, movilidad, árboles y otros

\usepackage{amssymb, amsmath} % Símbolos matemáticos
\usepackage{amsthm} % Teoremas, lemas, pruebas...
\usepackage{cancel} % Cancelar expresiones
\usepackage{multirow} % Tablas
\usepackage{multicol}
\usepackage{enumitem} % Enumerados a), b), c)... usando \begin{enumerate}[label=\alph*)]
\usepackage{xparse}
\usepackage[usenames]{color}
\usepackage[dvipsnames]{xcolor}
\usepackage{accents}
\usepackage{dsfont}


\usepackage{graphicx} % Inserción de imágenes
\definecolor{gray97}{gray}{.97}
\definecolor{gray75}{gray}{.75}
\definecolor{gray45}{gray}{.45}

\usepackage[hidelinks]{hyperref}  % Enlaces

\usepackage{listings} % Escribir código en diferentes lenguajes de programación
\usepackage{longtable} % para tablas largas

\title{Ejercicios de Análisis Matemático II}
\author{ }
\date{\today}

% Comandos
\newcommand{\enunciado}[1]{\textbf{#1}}
\newcommand{\revisado}[0]{ \includegraphics[width=0.04\textwidth]{./Imagenes/tick.jpeg}}
\newcommand{\enconstruccion}[0]{ \includegraphics[width=0.04\textwidth]{./Imagenes/construccion.png}}
\newcommand{\sinrevisar}[0]{ \includegraphics[width=0.055\textwidth]{./Imagenes/lupa.jpg}}
\newcommand{\espacio}{\vspace*{\baselineskip}}  % Añade espacios
\newcommand{\R}{\mathbb{R}}
\newcommand{\N}{\mathbb{N}}
\NewDocumentEnvironment{ejercicio}{mm}{\espacio\textbf{#1}\espacio}{\hfill\small{Hecho por \textit{#2}}}


% % % % % % % % % % % % % % % % % % % % % % % % % % % % % % % % %
%					 Inicio del documento
% % % % % % % % % % % % % % % % % % % % % % % % % % % % % % % % %
\begin{document}


\maketitle
\tableofcontents % Generando el indice
\newpage
\setlength\parindent{0pt} % Quitamos la sangría

\section{Sucesiones de funciones}
\subsection{Sucesiones de funciones}
	\subsection{Esquema para los tres primeros ejercicios}
	\input{./Ejercicios/Relacion1.1/esquema1-2-3.tex}
	\subsubsection{Ejercicio 1}
	\enunciado{Estudio la convergencia uniforme en intervalos de la forma $[0,a]$ y $[a,+\infty[$, donde $a>0$, de la sucesión de funciones $\{ f_n \}$ definidas para todo $x \geq 0$ por:}

\[f_n(x) = \frac{2nx^2}{1+n^2x^4}\]

$$f_n(x) = \frac{2nx^2}{1+n^2x^4} \hspace{2cm} f_n'(x) = \frac{4nx(1+n^2x^4)-8n^3x^5}{(1+n^2x^4)^2}$$

Tenemos que $f_n'(x) = 0 \iff 4nx+4n^3x^5-8n^3x^5 = 0 \iff x-n^2x^5 = 0 \iff x=\left\{
\begin{array}{l}
0, \\
\frac{1}{\sqrt n}
\end{array}\right.$

Comprobamos el signo de la derivada en medio de los intervalos con algún punto intermedio, sustituyendo:
$$f_n'\left(\frac{\sqrt 2}{\sqrt n}\right) < 0 \hspace{2cm} f_n'\left(\frac{1}{\sqrt{2n}}\right) > 0$$

Es decir, que la sucesión de máximos la vamos a poder coger como $x_n = f_n\left(\frac{1}{\sqrt n}\right)$.

Este máximo se encuentra solo en el primer conjunto que queremos estudiar, $[0,a]$, donde utilizando lo que hemos visto arriba, si estudiamos esta sucesión terminamos:

$$x_n = \frac{2}{2} = 1$$

Como no converge a 0, no hay convergencia uniforme en $[0,a]$. Por otro lado, en el conjunto $[a, +\inf)$ las $f_n$ son decrecientes, así que esta vez tomaremos la sucesión de máximos como $x_n = f_n(a)$. Pero aquí lo tenemos más fácil. En caso de haber estudiado que existe convergencia puntual en todo $\mathds{R^+_0}$, en concreto habríamos probado que hay convergencia puntual en $a$, por lo que $\{x_n\}$ converge a 0.

	\subsubsection{Ejercicio 2}
	\enunciado{Dado $\alpha \in \R$, consideremos la sucesión de funciones $\{f_n\}$, donde $\{f_n\}:[0,1] \longleftrightarrow \R$ es la función definida para todo $x\in[0,1]$ por:
$$f_n(x) = n^\alpha x(1-x^2)^n$$} 

El conjunto en el que tenemos que estudiar las funciones es el $[0,1]$, y las funciones son:

$$f_n(x) = n^\alpha x(1-x^2)^n$$ 
$$f_n'(x) = n^\alpha(1-x^2)^n - 2n^{\alpha+1}x^2(1-x^2)^{n-1} = n^\alpha(1-x^2)^{n-1}(1-x^2(1+2n))$$

Tenemos que $f_n '(x) = 0 \iff x=\left\{
\begin{array}{l}
1, \\
\frac{1}{\sqrt{2n+1}}
\end{array}\right.$

Vemos rápidamente que $f_n'(0) > 0$ por lo que las funciones son crecientes en la primera parte, llegan hasta $\frac{1}{\sqrt{2n+1}}$ y luego debe de bajar (porque $f_n(0) = f_n(1) = 0$), así que vamos a definir nuestra sucesión de máximos como $x_n = f_n\left(\frac{1}{\sqrt{2n+1}}\right)$. La estudiamos:

$$x_n = \frac{n^\alpha*(2n)^n}{(2n+1)^{n+\frac{1}{2}}} =  \frac{n^{n+\alpha}}{\sqrt 2(n+\frac{1}{2})^{n+\frac{1}{2}}} = \frac{1}{\sqrt 2}\frac{n^\alpha}{(n+\frac{1}{2})^{\frac{1}{2}}}\left(\frac{n}{n+\frac{1}{2}}\right)^n$$

La fracción de la derecha es equivalente a $e^{n\cdot ln\left(\frac{n}{n+\frac{1}{2}}\right)}$. Esto tiende a $\frac{1}{\sqrt e}$ cuando $n$ tiende a infinito. Usando que solo tenemos que estudiar este límite:

$$\lim\limits_{n->\infty} n\cdot ln\left(\frac{n}{n+\frac{1}{2}}\right)$$

Para hallar el límite estudiamos la siguiente función y aplicamos L'Hôpital:

$$g(x) = \frac{ln\left(\frac{n}{n+{\frac{1}{2}}}\right)}{\frac{1}{n}} = \frac{ln(n)-ln(n+1/2)}{\frac{1}{n}}$$



Sabiendo que la fracción de la derecha converge a $\frac{1}{\sqrt e}$ (habiendo hecho L'Hôpital arriba) y la fracción de la izquierda converge también, solo nos queda mirar la de en medio. Si $\alpha < 1/2$, esa sucesión converge a 0. Si $\alpha > 1/2$ esa sucesión diverge, y en el caso de $\alpha = 1/2$ esa sucesión converge a 1. Nos queda en total que ${x_n}$ converge a 0 $\iff \alpha < 1/2$. Por lo que hay convergencia uniforme solo en ese caso.

Para el caso en el que tenemos el conjunto $[\rho, 1]$ para algún $\rho > 1$, esta vez la funcion $f_n$ será decreciente en todo el intervalo a partir de un determinado $n$ (ya que el punto donde se alcanzaba el máximo antes se acerca a 0 conforme $n$ se va agrandando). Por lo que esta vez, a partir de ese $n_0$ con el que las $f_n$ con $n>n_0$ son decrecientes podemos coger la sucesión de máximos $x_n = f_n(\rho)$. Pero de nuevo, si hemos hecho la convergencia puntual, esta sucesión converge a 0, por lo que hay convergencia uniforme en este intervalo.

	\subsubsection{Ejercicio 3}
	\enunciado{Para cada $n\in\N$, sea $f_n:[0,\pi /2] \longleftrightarrow \R$ la función dada por: $$f_n(x) = n(cos(x))^nsen(x)$$}

El intervalo en el que nos piden la convergencia uniforme en el $[0, a]$ y en el $[a, \pi/2]$ de las siguientes funciones:

$$f_n(x) = n(cos(x))^nsen(x) \hspace{2cm} f_n'(x) = -n^2(cos(x))^{n-1}(sen(x))^2 + n(cos(x))^{n+1}$$

Sabemos que $cos(x) = 0 \iff x = \pi/2$ teniendo en cuenta que $x\in [0, \pi/2]$, momento en el que la derivada se hace cero. Si suponemos $cos(x) \neq 0$ podemos deducir que entonces $f_n'(x) = 0 \iff n(sen(x))^2 = cos(x)^2 \iff (tan(x))^2 = \frac{1}{n} \iff x = arctan\left(\frac{1}{\sqrt n}\right)$. Podemos notar que este x se va acercando cada vez más a 0 conforme avanza $n$. Este punto es un máximo de $f_n$ que podemos deducir de que $f_n'(0) > 0$ y $f_n'(\pi/4) < 0$ para todo $ n > 1$.

Definimos entonces nuestra sucesión de máximos de la función:

$$\{x_n\} = f_n\left(arctan\left(\frac{1}{\sqrt n}\right)\right) = n\left(cos\left(arctan\left(\frac{1}{\sqrt n}\right)\right)\right)^nsen\left(arctan\left(\frac{1}{\sqrt n}\right)\right)$$

Como podéis ver, esta sucesión es larguísima y no tiene pinta de tener una forma fácil de mejorarla, así que nos pasamos a alguna otra sucesión que siga los pasos de esta (esto solo nos vale si queremos buscar que es falsa la convergencia uniforme, que en este caso lo es para el conjunto $[0,a]$). Necesitamos una decreciente y que converja a 0, así que probamos con $y_n = f_n\left(\frac{1}{n}\right)$.

$$y_n = n\left(cos\left(\frac{1}{n}\right)\right)^nsen\left(\frac{1}{n}\right)$$

Sabemos que $\frac{sen(x)}{x}$ converge a 1 cuando x tiende a 0, que es el caso de $n\cdot sen\left(\frac{1}{n}\right)$. Nos quedaría $cos^n\left(\frac{1}{n}\right)$:

$$\lim\limits_{n->\infty} cos^n\left(\frac{1}{n}\right) = e^{\lim\limits_{n->\infty}n\cdot ln\left(cos\left(\frac{1}{n}\right)\right)}$$

Por L'Hôpital podemos sacar que el límite en el exponente es 0 (sale fácil si hacéis L'Hôpital en el $\lim\limits_{x->0}\frac{ln(cos(x))}{x}$ ), por lo que todo tiende a $e^0 = 1$, y por lo tanto, como la parte de $n\cdot sen\left(\frac{1}{n}\right)$ también converge a 1, podemos deducir que $y_n$ converge a 1. Hemos encontrado una sucesión de reales tales que $\{f_n(a_n)-f(n)\}$ no converge a 0, por lo que no hay convergencia uniforme en el $[0,a]$.

En el caso $[a, \pi/2]$ podemos cogernos la sucesión de números $x_n = f_n(a)$. Si bien antes no nos ha servido de nada el estudio del máximo, sí que podemos usar ahora que sabemos que existe un $n_0$ a partir del cual las funciones $f_n$ son decrecientes en el intervalo $[a, \pi/2]$ para todo $n>n_0$.

$$x_n = n(cos(a))^nsen(a)$$

Mismo razonamiento que en todos los ejercicios anteriores. Si hemos razonado ya que la convergencia puntual se da en todos los puntos del intervalo $[0, \pi/2]$, en concreto se da en $a$, por lo que $\{x_n\}$ converge a 0, y podemos deducir entonces que en este intervalo hay convergencia uniforme, usando el tercer punto que explicamos justo antes de los ejercicios.

 

	\subsubsection{Ejercicio 4}
	\begin{ejercicio}{%
Para cada $n \in \N$, sea $f_n:]0,\pi[ \to \R$ la función dada por:
\[f_n(x) = \frac{\sin^2(nx)}{n\sin(x)} \quad (0 < x < \pi)\]
Estudia la convergencia puntual de la sucesión puntual $\{f_n\}$ así como la
convergencia uniforme en intervalos del tipo $]0,a], [a,\pi[$ y $[a,b]$ donde $0 < a< b < \pi$.
}{Pablo Baeyens}

\espacio

\noindent\textbf{$\{f_n\}$  converge puntualmente a 0}

Sea $x \in (0,\pi)$. Para todo $n \in \N$ tenemos:
\[ 0 \leq \frac{\sin^2(nx)}{n\sin(x)} \leq \frac{1}{n\sin(x)} \]
Por tanto, como $\{\frac{1}{n\sin(x)}\} \to 0$, por el teorema del sándwich, la
sucesión converge a 0.

\espacio

\noindent\textbf{La sucesión no converge uniformemente en intervalos de la forma $(0,a]$}

La convergencia uniforme es equivalente a que para cualquier sucesión $\{a_n\}$:

\[\{f_n(a_n) -f(a_n)\} = \left\{\frac{\sin^2(na_n)}{n\sin(a_n)}\right\} \to 0\]

Sea $a_n = \frac{1}{n}$:

\[ \left\{\frac{\sin^2(na_n)}{n\sin(a_n)}\right\}  =
\left\{\frac{\sin^2(1)}{\frac{\sin(\frac{1}{n})}{\frac{1}{n}}}\right\} \to \sin^2(1) \neq 0 \]

\espacio

\noindent\textbf{La sucesión no converge uniformemente en intervalos de la forma $[a,\pi)$}

Sea $a_n = \pi - \frac{1}{n}$:

\[ \left\{\frac{\sin^2(na_n)}{n\sin(a_n)}\right\}  =
\left\{\frac{\sin^2(n\pi -1)}{\frac{\sin(\pi -\frac{1}{n})}{\frac{1}{n}}}\right\} =
\left\{\frac{\sin^2(1)}{\frac{\sin(\frac{1}{n})}{\frac{1}{n}}}\right\} \to \sin^2(1) \neq 0 \]

Ya que $\sin(x) = \sin(\pi-x)$.

\espacio

\noindent\textbf{La sucesión converge uniformemente en intervalos de la forma $[a,b]$}

Por el teorema de Weierstrass
$\exists M >0: \forall x \in [a,b]: \left|\frac{1}{\sin(x)}\right| \leq M$. Por tanto:

\[ \left|\frac{\sin^2(nx)}{n\sin(x)}\right| \leq \frac{M}{n} \]

Como $\{\frac{M}{n}\} \to 0$, la sucesión converge uniformemente en intervalos
de la forma $[a,b]$.

\end{ejercicio}
 

	%\subsubsection{Ejercicio 5}
	%\input{./Ejercicios/Relacion1.1/Rel_1.1_ej5.tex}
	%\subsubsection{Ejercicio 6}	
	%\input{./Ejercicios/Relacion1.1/Rel_1.1_ej6.tex}
	%\subsubsection{Ejercicio 7}	
	%\input{./Ejercicios/Relacion1.1/Rel_1.1_ej7.tex}
	%\subsubsection{Ejercicio 8}	
	%\enunciado{
Sea, para cada $n\in\mathds{N}$,
$$f_n(x) = \frac{x}{n^\alpha(1+nx^2)} \hspace{0.5cm} (x\geq0)$$
Prueba que la serie $\sum f_n$ converge}

\textit{a)} puntualmente en $\mathds{R}^+_0$ si $\alpha > 0$
\textit{b)} uniformemente en semirrectas cerradas que no contienen al cero
\textit{c)} uniformemente en $\mathds{R}^+_0$ si $\alpha > 1/2$

\textit{a)} 
Podemos escribir las funciones $f_n$ como:

$$f_n(x) = \frac{x}{n^\alpha+n^{\alpha+1}x^2} $$

Ahora, fijado un $x\in\mathds{R^+}$, usamos el criterio de comparación por paso al límite con la serie $\frac{1}{n^{\alpha+1}}$, que sabemos que converge a un número real, puesto que el exponente de la $n$ en el denominador es mayor estricto que 1:

$$\lim\limits_{n->\infty} \frac{\frac{x}{n^\alpha+n^{\alpha+1}x^2}}{\frac{1}{n^{\alpha+1}}} = \lim\limits_{n->\infty} \frac{x}{\frac{1}{n}+x^2} = \frac{1}{x} $$

Como $1/x$ es un número real distinto de 0, por el criterio sabemos que nuestra serie con los términos $f_n(x)$ también converge (le pasa lo mismo que con la que la hemos comparado).

Para el caso en que $x$ sea 0, es fácil ver que la serie converge puntualmente, puesto que vale 0.


\textit{b)}  Sea la semirrecta $[c, \infty)$, con $c > 0$ podemos ver que:

$$f_n(x) = \frac{x}{n^\alpha+n^{\alpha+1}x^2} \leq \frac{x}{n^{\alpha+1}x^2} = \frac{1}{n^{\alpha+1}x} \leq \frac{1}{n^{\alpha+1}c}$$
Esta última serie es independiente de las $x\in [c, \infty)$ y converge (se puede comprobar por comparación paso al límite con $\frac{1}{n^{\alpha+1}}$ el límite es $c$, que es una constante mayor que 0). Por el test de Weierstrass, podemos afirmar que $f_n$ converge uniformemente en toda la semirrecta $[c, \infty)$.


\textit{c)} Hacemos las derivadas de $f_n$:

$$f_n'(x) = \frac{n^\alpha + n^{\alpha+1}x^2-2x^2n^{\alpha+1}}{(n^\alpha+n^{\alpha+1}x^2)^2} = \frac{n^\alpha(1 - nx^2)}{(n^\alpha+n^{\alpha+1}x^2)^2}$$

Las funciones $f_n$ son crecientes hasta $x=\frac{1}{\sqrt{n}}$, donde empiezan a decrecer, así que esta abscisa es un máximo de las $f_n$. De nuevo, podemos acotar la serie por una sucesión de números reales $a_n$ y concluir que covergen uniformemente en $\mathds{R}^+_0$. Definimos la sucesión de los máximos de la función: $a_n = f_n\left(\frac{1}{\sqrt{n}}\right)$. Vemos que:

$$f_n(x) \leq a_n = f_n\left(\frac{1}{\sqrt{n}}\right) = \frac{\frac{1}{\sqrt{n}}}{n^\alpha(1+1)} = \frac{1}{2n^{\alpha+1/2}} \hspace{0.5cm} \forall x \in \mathds{R}^+_0$$
La serie de términos $a_n$ converge ya que $(\alpha + 1/2) > 1$, así que por el test de Weierstrass, la serie $f_n$ converge uniformemente en $\mathds{R}^+_0$.
	
	%\subsubsection{Ejercicio 9}
	%\input{./Ejercicios/Relacion1.1/Rel_1.1_ej9.tex}
	%\subsubsection{Ejercicio 10}
	%\input{./Ejercicios/Relacion1.1/Rel_1.1_ej10.tex}
	%\subsubsection{Ejercicio 11}
	%\input{./Ejercicios/Relacion1.1/Rel_1.1_ej11.tex}
	%\subsubsection{Ejercicio 12}
	%\input{./Ejercicios/Relacion1.1/Rel_1.1_ej12.tex}
	
\subsection{Series de potencias}
	%\subsubsection{Ejercicio 1}
	%\input{./Ejercicios/Relacion1.2/Rel_1.2_ej1.tex}
	%\subsubsection{Ejercicio 2}
	%\input{./Ejercicios/Relacion1.2/Rel_1.2_ej2.tex}
	%\subsubsection{Ejercicio 3}
	%\input{./Ejercicios/Relacion1.2/Rel_1.2_ej3.tex}
	%\subsubsection{Ejercicio 4}
	%\input{./Ejercicios/Relacion1.2/Rel_1.2_ej4.tex}
	%\subsubsection{Ejercicio 5}
	%\input{./Ejercicios/Relacion1.2/Rel_1.2_ej5.tex}
	%\subsubsection{Ejercicio 6}
	%\input{./Ejercicios/Relacion1.2/Rel_1.2_ej6.tex}
	%\subsubsection{Ejercicio 7}
	%\input{./Ejercicios/Relacion1.2/Rel_1.2_ej7.tex}
	%\subsubsection{Ejercicio 8}
	%\input{./Ejercicios/Relacion1.2/Rel_1.2_ej8.tex}
	%\subsubsection{Ejercicio 9}
	%\input{./Ejercicios/Relacion1.2/Rel_1.2_ej9.tex}
	
\newpage
\section{Integral de Lebesgue}
\subsection{Medida de Lebesgue en $\mathbb{R}^N$}
	%\subsubsection{Ejercicio 1}
	%\input{./Ejercicios/Relacion2.1/Rel_2.1_ej1.tex}
	%\subsubsection{Ejercicio 2}
	%\input{./Ejercicios/Relacion2.1/Rel_2.1_ej2.tex}
	\subsubsection{Ejercicio 3}
	\enunciado{Sea $(\Omega ,\mathcal{A})$ un espacio medible y sea $\Omega '$ un nuevo conjunto. Sea $f: \Omega \longrightarrow \Omega '$ una aplicación. Probar que $( \Omega ', \{ B\in\Omega ':\ f^{-1}(B)\in \mathcal{A} \} )$ es un espacio medible.}

Definimos $\mathcal{A}'=\{ B\in\Omega ':\ f^{-1}(B)\in \mathcal{A} \}$, debemos demostrar que es $\sigma$-álgebra.

\textit{i)}
$\Omega ' \in \mathcal{A}'$:

$f^{-1}(\Omega ') = \Omega\in\mathcal{A}$ por ser $\mathcal{A}$ $\sigma$-álgebra.

\textit{ii)}
Si $\{A_n\}$ es una sucesión de elementos de $\mathcal{A}'$, entonces $\left( \bigcup_{n\in\mathbb{N}}A_n \right) \in \mathcal{A}'$:

Si $A_i\in \mathcal{A}' \ \forall       
 i\in\mathbb{N} \implies$
$\forall i\in\mathbb{N} A_i\subseteq\Omega '\ : \left( f^{-1}(A_i)\right) \in \mathcal{A}$

\[ \bigcup_{n \in \mathbb{N} }f^{-1}(A_n) \in \mathcal{A} \ por \ ser \ \sigma -álgebra\] 

\textit{iii)}
Si $A\in\mathcal{A}'$ entonces $A^c=\Omega '\setminus A \in \mathcal{A}'$:

Si $A\in \mathcal{A}' \implies f^{-1}(A)\in \mathcal{A}$

\[f^{-1}(A^c) = f^{-1}(\Omega'\setminus A) = f^{-1}(\Omega') - f^{-1}(A) = \Omega - f^{-1}(A) \]

Por ser $\Omega , f^{-1}(A)\in \mathcal{A} \implies f^{-1}(A^c)\in \mathcal{A}$ por ser $\sigma$-álgebra. Por tanto $A^c\in \mathcal{A}$

$\Longrightarrow (\Omega', \mathcal{A}')$ es un espacio medible. 
	\subsubsection{Ejercicio 4}
	 \enunciado{Sea $\mu^*$ una medida exterior en un conjunto no vacío $\Omega$. Probar que la restricci\'on de $\mu^*$ a la $\sigma$  -álgebra $ C_{\Omega, \mu^*}$ es una medida completa ( esto es, todo subconjunto B de un conjunto $Z \in   C_{\Omega, \mu^*}$ tal que $ \mu^* (Z) = 0$ es también un conjunto de la propia $\sigma$ -álgebra)}

Sea Z un conjunto tal que $\mu^* (Z) = 0 $ y $B \subseteq Z$ , entonces , B ser\'a  medible si y solo si 

\[\forall A \subseteq \Omega , \mu^* (A) = \mu^* (A\cap B) + \mu^* (A \cap  B^c).\]

Para probarlo partimos de la desigualdad que nos da la subaditividad de una medida exterior, es decir:
\[A \subseteq \Omega \; \mu^* (A) \leq \mu^* (A \cap B) + \mu^* (A\cap B^c)\]
Si además usamos que\: 
\[A \cap B \subseteq A \cap  B \subseteq Z \Rightarrow  \mu^* (A \cap  B) = 0\]
 entonces obtenemos 
\[A \subseteq \Omega \; \mu^* (A) \leq \mu^* (A \cap  B) + \mu^* (A\cap B^c) \leq  \mu^* (A)\] como queríamos demostrar. 
	\subsubsection{Ejercicio 5}
	\enunciado{Probar que M es la mayor $\sigma$ -\'algebra que contiene los intervalos acotados y sobre la que $\lambda ^*$ es aditiva.}

Al estar hablando de intervalos $\Omega = \mathbb{R}$
Supongamos que existe otra $\sigma$ -\'algebra $N$ que contiene los intervalos acotados y sobre la que $\lambda ^*$ es aditiva. Terminaremos demostrando que en ese caso $N \subseteq M$.

Recordemos que $M=\{B \cup Z : B\in \mathfrak{B}, \lambda ^*(Z)=0\} \subseteq C_{\mathbb{R},\lambda }$

Llamaremos $\lambda '=\lambda	^* / N$

Cogemos un conjunto arbitrario $A\subseteq \mathbb{R}$.
Sabemos, en virtud de la propiedad de regularidad de la medida exterior (Prop. 2.1.10), que existe un boreliano $B$

\[ B:A\subseteq B, \lambda '(A)=\lambda '(B) \]

Usando la propiedad de que $N$ contiene los intervalos acotados podemos usar la $\sigma-$aditividad. Sea $E\subseteq \mathbb{R}$ perteneciente a los intervalos acotados. Vemos que $B\cap E, B\cap E^c \in N$ 

$\lambda '(B) = \lambda'\left( (B\cap E) \cup (B\cap E^c) \right) $
$\geqslant\lambda '(A\cap E) + \lambda'(A\cap E^c) \geqslant \lambda '(A)$

Por tanto $E \in C_{\mathbb{R},\lambda '}$ lo que es equivalente (cuando $\Omega = \mathbb{R}^N con N\in\mathbb{N}$) a: $E \in M$

$\implies N \subseteq M$
 

	\subsubsection{Ejercicio 6}
	\enunciado{Probar que la uni\'on numerable de conjuntos de medida nula es un conjunto de medida nula.
          Ded\'uzcase que $\mathbb{N}$ y $\mathbb{Q}$ son dos conjuntos de medida nula.}\\

  Sea $\mu(A_i)=0 \ \forall i \in \mathbb{N}$. Entonces, por la $\sigma$-aditividad, \
  $\mu(\bigcup_{n=1}^{\infty} A_n) = \sum_{n=1}^{\infty} \mu(A_n) = 0$.\\

  Probemos ahora que $\mathbb{N}$ es un conjunto de medida nula.
  Para ello, demostraremos en primer lugar que
  $\mu(\lbrace n \rbrace) = 0 \ \forall n \in \mathbb{N}$.
  Definimos $A_k = [n - \frac{1}{k}, n + \frac{1}{k}]$.
  Por tanto, $\mu(\bigcap_{n \in \mathbb{N}} A_k) = \lim_k\mu(A_k) = \lim_k\frac{2}{k} = 0$.
  Demostramos as\'i que
  $\mu(\mathbb{N}) = \mu(\bigcup_{n \in \mathbb{N}} \lbrace n \rbrace)
  = \sum_{n=1}^{\infty} \mu(\lbrace n \rbrace) = 0$.\\

  La prueba para $\mathbb{Q}$ es muy parecida.
  Como $\mathbb{Q}$ es numerable, podemos definir una sucesi\'on $\lbrace q_n \rbrace$ tal que
  $q_n \in \mathbb{Q} \ \forall n \in \mathbb{N}$ y
  $\bigcup_{n \in \mathbb{N}} \lbrace q_n \rbrace = \mathbb{Q}$. Por tanto,
  $\mu(\mathbb{Q}) = \mu(\bigcup_{n \in \mathbb{N}} \lbrace q_n \rbrace)
  = \sum_{n=1}^{\infty} \mu(\lbrace q_n \rbrace) = 0$.

	\subsubsection{Ejercicio 7}
	\enunciado{7. Existencia de conjuntos no medibles}

\begin{enumerate}[label=\alph*)]
	\item \textbf{Probar que la familia $\{x + \mathbb Q : x \in \mathbb R \}$ es una partición de $\mathbb R$.} \\
	
	Sea $x \in \mathbb R$. Entonces $x \in x+\mathbb Q$ dado que $x = x + 0$. Por ello, $\displaystyle \bigcup_{x \in \mathbb R} \{x+\mathbb Q\} = \mathbb R$.
	
	Sean $x,y,t \in \mathbb R : t \in x+\mathbb Q$ y $t \in y + \mathbb Q$. Entonces $\exists q_1, q_2 \in \mathbb Q : t = x + q_1 = y + q_2$. Así, $x = y + q_2 - q_1$, y, como $q_2 - q_1 \in \mathbb Q$, $x \in y + \mathbb Q$ y $x + \mathbb Q = y + \mathbb Q$.
	
	Así, esta familia está formada por conjuntos disjuntos (si un número está en dos elementos de la familia, estos son el mismo) cuya unión es $\mathbb R$: es una partición de $\mathbb R$.
	
	\item \textbf{Pongamos $\{x+\mathbb Q : x \in \mathbb R\} = \{A_i : i \in I\} \ (A_i \ne A_j$ para $i \ne j)$ y, para cada $i \in I$, sea $x_i \in A_i \ \cap \ ]0, 1]$. Probar que el conjunto $E = \{x_i : i \in I\}$ no es medible.} \\
	
	Sea $\{q_n : n \in \mathbb N\}$ una numeración de $]-1, 1] \cap \mathbb Q$. Obsérvese que $\{q_n + E : n \in \mathbb N\}$ es una familia de conjuntos disjuntos entre sí: si hay un $x \in (q_a + E) \cap (q_b + E)$ para algunos $a, b \in \mathbb N$, entonces $x = q_a + x_i = q_b + x_j$ para algunos $i,j \in I$. En ese caso $x_i = x_j + q_b - q_a \implies x_i \in x_j + \mathbb Q$. Como cada $x_i$ está escogido de forma que dos distintos pertenecen a un elemento distinto de la partición $\{x + \mathbb Q : x \in \mathbb R\}$, necesariamente $x_i = x_j \implies q_a = q_b \implies q_a + E = q_b + E$.
	
	Supongamos que $E$ es medible. En tal caso, $\lambda(E) = \lambda(E+k) \ \forall k \in \mathbb R$ dado que $\lambda$ es invariante por traslación. Debido a la $\sigma$-aditividad de $\lambda$ y a que los conjuntos $q_n + E$ son disjuntos entre sí, resulta que:
	
	$$\lambda(\bigcup_{n \in \mathbb N} (q_n + E)) = \sum_{n \in \mathbb N}\lambda(q_n + E) = \sum_{n \in \mathbb N} \lambda(E)$$
	
	Puede verse que $\displaystyle ]0, 1] \subseteq \bigcup_{n \in \mathbb N} (q_n + E) \subseteq \ ]-1, 2]$. Demostración de la primera inclusión: sea $x \in \ ]0, 1]$. $x \in x + \mathbb Q \implies \exists i \in I : x \in A_i \implies \exists i \in I \exists q \in \mathbb Q : x = x_i + q \implies \exists q \in \mathbb Q : x \in q + E$, y como $|x_i-x| < 1$ puesto que $x_i, x \in \ ]0,1]$, ocurre que $q \in \ ]-1,1]$. Así, $q = q_k$ para algún $k \in \mathbb N$ y $\displaystyle x \in \bigcup_{n \in \mathbb N} (q_n + E)$. La otra inclusión se debe al rango de valores posible de los $q_n$ y los $x_i$.
	
	Aplicando que $A \subseteq B \implies \lambda(A) \le \lambda(B)$, tendremos que $\displaystyle \lambda(]0, 1]) = 1 \le \lambda(\bigcup_{n \in \mathbb N} (q_n + E)) = \sum_{n \in \mathbb N} \lambda(E) \le \lambda(]-1, 2]) = 3$. Como esto es imposible tanto si $\lambda(E) = 0$ (en cuyo caso $\displaystyle \sum_{n \in \mathbb N} \lambda(E) = 0 \ngeq 1$) como si $\lambda(E) \in \mathbb R^+$ (en cuyo caso $\displaystyle \sum_{n \in \mathbb N} \lambda(E) = +\infty \nleq 3$), la suposición de que $E$ es medible resulta haber sido incorrecta, y $E$ no es medible.
	
	\item \textbf{Probar que cualquier subconjunto medible de $E$ tiene medida cero.} \\
	
	Los conjuntos $q_n + A$ son, de nuevo, disjuntos. Por ello, vuelve a ocurrir que $\displaystyle \lambda(\bigcup_{n \in \mathbb N} (q_n + A)) = \sum_{n \in \mathbb N} \lambda(A)$. De nuevo, $\displaystyle \bigcup_{n \in \mathbb N} (q_n + A) \subseteq \ ]-1, 2]$ y consecuentemente $\displaystyle \sum_{n \in \mathbb N} \lambda(A) \le \lambda(]-1, 2]) = 3$. La única posibilidad es que $\lambda(A) = 0$.
	
	\item \textbf{Sea $M \subseteq \mathbb R$ con $\lambda^*(M) > 0$. Probar que $M$ contiene un subconjunto no medible.} \\
	
	Si $M$ no es medible, el enunciado es cierto ($M$ sería un subconjunto no medible de $M$). Sea $M$ medible, es decir, $\lambda(M) = \lambda^*(M)$. Se observa que $\displaystyle M = \bigcup_{q \in \mathbb Q} M \cap (q + E)$: todo $m \in M$ tendrá en $E$ un $x_i$ representante de su clase de equivalencia según la partición $\{x + Q : x \in \mathbb R\}$, por lo que $x = q_a + x_i \in q_a + E$.
	
	Supongamos que $M\cap (q + E)$ es medible para todo $q \in \mathbb Q$. En ese caso: (aparece un $\le$ porque la unión no es disjunta)
	$$\displaystyle \lambda(M) = \lambda(\bigcup_{q \in \mathbb Q} M \cap (q + E)) \le \sum_{q \in \mathbb Q} \lambda(M \cap (q+E))  = \sum_{q \in \mathbb Q} \lambda((M-q) \cap E)$$
	
	Que es igual a $0$ por ser la suma de las medidas de subconjuntos de $E$ medibles (porque suponemos que todos son medibles), las cuales son $0$ por lo probado en $c)$. Contradicción (hemos obtenido que $\lambda(M) \le 0$), por lo cual alguno de los $M\cap (q + E)$ no será medible.
\end{enumerate}
 

	\subsubsection{Ejercicio 8}
	\enunciado{Probar que la existencia de conjuntos no medibles equivale a la no aditividad de $\lambda^*$.}

 
Sabemos que $E\in M \Longleftrightarrow E \in C_{\mathbb{R}, \lambda^*}$

Queremos probar:
$E \in M \ no \ medible \Longleftrightarrow \lambda^*(A) \geq \lambda^*(A\cap E) + \lambda^*(A \cap E^c) \ \forall A\subset \mathbb{R}^N$

$\Longrightarrow$

Si $E \in \mathbb{R}^N no es medible$, entonces existe $A\subseteq \mathbb{R}^N$ tal que 
\[\lambda^*( (A\cap E) \cup (A\cap E^c) ) = \lambda^*(A) < \lambda^*(A\cap E) + \lambda^*(A\cap E^c)\]
y $\lambda^*$ no es aditiva.

$\Longleftarrow$

Supongamos que $\lambda^*$ no es aditiva. Sean $A,B\subseteq \mathbb{R}^N, \ A\cap B = \varnothing$ tales que

\[\lambda^*(A\cup B) < \lambda^*(A) + \lambda^*(B)\]

Se tiene que
\[ \lambda^*(A\cup B) < \lambda^*( (A\cup B)\cap A ) + \lambda^*( (A\cup B)\cap A^c )\]
y el conjunto $A$ no es medible.
	\subsubsection{Ejercicio 9}
	\enunciado{Sean $A$ un abierto de $\mathbb{R}^N$ y $f:A \rightarrow \mathbb{R}^M$ una función de clase $C^1$ con $N<M$. Probar que $f(A)$ es de medida cero.}

Sea $G$ un abierto de $\mathbb{R}^N$ y sea $f \in C^1(G)$

1. Si $Z \subseteq G(=B \supseteq Ax\{0\})$, $\lambda (Z) = 0$, entonces $\lambda (f(Z)) = 0$

Definimos $B := A x \mathbb{R}^{M-N} \subseteq \mathbb{R}^M$ y $g: B \rightarrow \mathbb{R}^M$ por
\[ g(x,y) = f(x) \ \forall (x,y)\in B\]
El conjunto $B$ es un abierto de $\mathbb{R}^M$ y $g\in C^1(B)$

El conjunto $Ax\{0\} \subset B$ es de medida cero en $\mathbb{R}^M$ por estar contenido en un hiperplano.

\[\lambda(Ax\{0\}) = \lambda (A) \cdot \lambda (0) = 0\]

Aplicando la proposición anterior $\lambda( g(A,\{0\}) ) = 0 \implies \lambda (f(A)) = 0$
	%\subsubsection{Ejercicio 10}
	%\input{./Ejercicios/Relacion2.1/Rel_2.1_ej10.tex}
	%\subsubsection{Ejercicio 11}
	%\input{./Ejercicios/Relacion2.1/Rel_2.1_ej11.tex}
	
\subsection{Integral de Lebesgue en $\mathbb{R}^N$}
	%\subsubsection{Ejercicio 1}
	%\enunciado{Sean $(\Omega, \mathcal{A})$ un espacio medible y sea $E\in\mathcal{A}$.} 

\textit{a)} 
Si $f:\Omega \rightarrow \mathbb{R}$ es una función medible entonces $f|_E$ es una función medible cuando se considera el espacio medible $(E,\mathcal{A}_E)$.

$$\mathcal{A}_E = \{ A\cap E: A\in\mathcal{A} \}$$
$f$ medible $\implies f^{-1}(B)\in\mathcal{A} \hspace{1cm} \forall B \in \mathbb{B}$ 

$E\subseteq\Omega \implies f^{-1}_E(B)= f^{-1}(B)\cap E \in \mathcal{A}_E \hspace{1cm} \forall B\in\mathcal{B}$
$\implies f|_E$ es medible en $(E, \mathcal{A}_E)$

\ 

\textit{b)} Sea $f:E\rightarrow\mathbb{R}$ una función. Probar que $f$ es medible si, y sólo si, la función $f_{\mathcal{X} E}:\Omega\rightarrow\mathbb{R}$, definida por
\[ f_{\mathcal{X} E}(x) = \left\{
	\begin{matrix}
		f(x)			& \mbox{si } x\in E
		\\0			& \mbox{si } x\not\in E
	\end{matrix} \right.
\]
es medible. Además si $\Omega = \mathbb{R}^N$ entonces
\[ \int_Efd\lambda = \int_{\mathbb{R}^N} f \mathcal{X}E d\lambda\]

$\Longleftarrow (f_{\mathcal{X}E} \ medible \implies f \ medible)$

Por el apartado anterior $f_{\mathcal{X}E}|_E = f$ es medible.

$\vspace{0.5cm}$
$\Longrightarrow (f\ medible \implies f_{\mathcal{X}E} \ medible)$

\[ f^{-1}_{\mathcal{X}E}(B) = \left\{
	\begin{matrix}
		f^{-1}(B)\cap E & \mbox{si } 0\not\in B
		\\ E^c\cup ( f^{-1}(B)\cap E ) & \mbox{si } 0\in B 
	\end{matrix} \right.	
\] 
Por las propiedades de las $\sigma$-álgebra y los espacios medibles:

$f^{-1}(B)\cap E \in \mathcal{A}$ y $E^c\cup ( f^{-1}(B)\cap E )\in \mathcal{A} \implies f^{-1}_{\mathcal{X}E}(B) \in \mathcal{A}$

\

Como $\Omega = \mathbb{R}^N$

$\int_{\mathbb{R}^N}f_{\mathcal{X}E}d\lambda  = \int_Efd\lambda + \int_{E^c}0d\lambda$
$= \int_Efd\lambda + \lambda(R(0)) = \int_Efd\lambda$

\

\textit{c)} Sea $f:E\rightarrow\mathbb{R}$ una función medible. Probar que $f$ es integrable en $E$ si, y sólo si, $f_{\mathcal{X}E}$ es integrable en $\mathbb{R}^N$

Probando que $|f|_{\mathcal{X}E} = |f_{\mathcal{X}E}|$ por el apartado anterior sabemos que $\int|f|d\lambda = \int|f_{\mathcal{X}E}|d\lambda	$

\[ |f|_{\mathcal{X}E}(x) = \left\{
	\begin{matrix}
		|f(x)|	 & \mbox{si } x\in E
		\\	0	 & \mbox{si } x\not\in E	
	\end{matrix} \right.
	= \left\{
	\begin{matrix}
		f(x) 	& \mbox{si } f(x)>0
		\\-f(x) & \mbox{si } f(x)<0
		\\ 0		& \mbox{si } x\not\in E \mbox{ ó } f(x)=0
	\end{matrix} \right.
\]

\[ |f_{\mathcal{X}E}(x)| = \left\{
	\begin{matrix}
		f_{\mathcal{X}E}(x)  & \mbox{si } f_{\mathcal{X}E}>0
		\\ -f_{\mathcal{X}E}(x)  & \mbox{si } f_{\mathcal{X}E}<0
		\\ 0  & \mbox{si } f_{\mathcal{X}E}=0
	\end{matrix} \right.
	= \left\{
	\begin{matrix}
		f(x) 	& \mbox{si } f(x)>0
		\\-f(x) & \mbox{si } f(x)<0
		\\ 0		& \mbox{si } x\not\in E \mbox{ ó } f(x)=0
	\end{matrix} \right.
\]
	%\subsubsection{Ejercicio 2}
	%\input{./Ejercicios/Relacion2.2/Rel_2.2_ej2.tex}
	%\subsubsection{Ejercicio 3}
	%\input{./Ejercicios/Relacion2.2/Rel_2.2_ej3.tex}
	%\subsubsection{Ejercicio 4}
	%\input{./Ejercicios/Relacion2.2/Rel_2.2_ej4.tex}
	%\subsubsection{Ejercicio 5}
	%\input{./Ejercicios/Relacion2.2/Rel_2.2_ej5.tex}
	%\subsubsection{Ejercicio 6}
	%\input{./Ejercicios/Relacion2.2/Rel_2.2_ej6.tex}
	
\subsection{Teoremas de convergencia}
	%\subsubsection{Ejercicio 1}
	%\enunciado{Sea $(\Omega, \mathcal{A})$ un espacio de medida, $\{f_n\}$ una sucesión de funciones medibles y $f,g$ dos funciones medibles. Porbar las siguientes afiirmaciones:}


\textit{a)}
	Si $\{f_n\}$ converge c.p.d. a $f$ y a $g$ c.p.d. entonces $f=g$ c.p.d. 

Definimos
\[A = \{ x\in\Omega / \{f_n(x)\} \rightarrow f(x) \} \hspace{0.5cm}
  B = \{ x\in\Omega / \{f_n(x)\} \rightarrow g(x) \} \hspace{0.5cm}
\]

\[  D = \{ x\in\Omega / f(x) = g(x) \} 
\]
Sabemos por hipótesis que $\mu (A^c) = \mu (B^c) = 0$ y si $x\in (A\cap B) \implies f(x) = g(x)$.

\[D^c \subseteq (A\cap B)^c = A^c \cup B^c \]
Suponemos que $D^c$ es medible, por tanto
$\mu (D^c) \leq \mu (A^c \cup B^c) \leq \mu(A^c) + \mu (B^c) = 0$

Por tanto $\mu (D^c)=0$ y $f=g$ c.p.d. 

{\ } 
 
\textit{b)}
	Si $\{f_n\}$ converge c.p.d. a $f$ y $f=g$ c.p.d. entonces $\{f_n\}$ converge c.p.d. a $g$.
	
Con los conjuntos anteriores sabemos que $\mu (A^c) = 0$, $\mu (D^c) = 0$ y en este caso
\[Si \ x\in D, entonces \ x\in B \implies \mu (B^c) \leq \mu (D^c) = 0\]
Por tanto $\{f_n\}$ converge c.p.d. a $g$.

	%\subsubsection{Ejercicio 2}
	%\enunciado{Considerar las siguientes sucesiones $\{f_n\}$ de funciones reales de variable real:}
	\[f_n = \frac{1}{n}\mathcal{X}_{[-n, n]} \hspace{0.5cm} g_n = n^2\mathcal{X}_{[1/(n+1), 1/n]}\]


Estudiar en cada caso la convergencia puntual y comparar $\int lim_{n \rightarrow \infty} f_n$
y $lim_{n \rightarrow \infty} \int f_n$ 

\textit{a)}  $f_n = \frac{1}{n}\mathcal{X}_{[-n, n]}$


$|f_n(x)| \leq \frac{1}{n} \ \forall n\in\mathbb{N}$ por tanto $\{f_n\}$ converge uniformemente a $0$ en $\mathbb{R}$.

$f_n$ es continua c.p.d. y acotada.
\[ \forall n\in \mathbb{N} \hspace{0.5cm} \int_{\R} f_n = 
	\int_{-\infty}^{-n} f_n + \int_{-n}^n f_n + \int_n^{+\infty} f_n = 0 + \int_{-n}^n f_n + 0
\] 
\[ \int_{\R} |f_n| \leq \int_{\R} \frac{1}{n} \mathcal{X}_{[-n, n]} = 
	\frac{1}{n}\int_{\R} \mathcal{X}_{[-n, n]} = \frac{2n}{n} = 2
\]
\[lim_{n\rightarrow \infty} \int f_n = 2 \not = \int lim_{n\rightarrow \infty} f_n = \int 0 = 0\]

{\ }

\textit{b)} $g_n = n^2\mathcal{X}_{[1/(n+1), 1/n]}$

\[\R = ]-\infty, 0[ \cup [0, 5[ \cup [5, +\infty[ \hspace{1cm}
 \forall x\in \left[ \frac{1}{n+1}, \frac{1}{n} \right] \subseteq [0,5[ \hspace{0.5cm} 		
 \{g_n(x)\}\rightarrow 0
\]
\[ x\in[0, 5[ \implies \exists n_o :  x\not\in [0, 5[ \hspace{0.5cm} \forall n\geq m
\]
\[ lim_{n\rightarrow\infty} g_n = 0 \implies \int_{\R} lim_{n\rightarrow\infty} g_n = 0
\]
Comprobamos ahora el límite de la integral
\[ \forall n\in\N \int_{\R} n^2\mathcal{X}_{[\frac{1}{n+1}, \frac{1}{n}]} =
	n^2 \int_{\R}\mathcal{X}_{[\frac{1}{n+1}, \frac{1}{n}]} = \frac{n}{n+1}
\]
Como $lim_{n\rightarrow\infty} \frac{n}{n+1} = 1$, entonces $lim_{n\rightarrow\infty}\int g_n = 1$
	%\subsubsection{Ejercicio 3}
	%\enunciado{Sea $E\subseteq \R ^n$ un conjunto medible de medida finita, y sea $\{f_n\}$ una sucesión de funciones medibles en $E$ que converge puntualmente a una función $f$. Supongmos que existe una constante $M\geq 0$ tal que $|\{ f_n \}| \leq M \ \forall n\in\N$. Probar que f es integrable y que
\[ lim\int_E |f-f_n| = 0 \]
Dar un ejemplo mostrando que la hipótesis de medida finita no puede ser suprimida. 
} 

Si $\lambda (E) < +\infty$
\[ \int_E |f| \leq \int_E M = M\lambda (E) < +\infty\]
y por definición $f$ es integrable.

$\{f_n\}$ es una sucesión de funciones medibles que converge puntualmente a $f$ en $E$ y existe una función $g$ integrable en $E$ tal que $\{ |f_n| \}\leq g \ \forall x\in E$. 
Podemos tomar como función $g$ la función constantemente igual a $M$, $g$ es integrable en $E$ ya que $\lambda (E)<+\infty$ (si quitamos la condición de medida finita $g$ no sería integrable).

Por el teorema de la convergencia dominada
\[ lim \int_E |f-f_n| = 0
\]

Si quitamos la condición $\lambda (E)<+\infty$ y cogemos $\{f_n\}$ sucesión de funciones tal que $f_n(x) = 1 \ \forall x \in E$, vemos que $f=lim_{n\rightarrow \infty} f_n$ es una función no integrable, ya que
\[\int_Ef = \int_E1 = \lambda (E) = +\infty \not < +\infty \]
	%\subsubsection{Ejercicio 4}
	%\enunciado{ Caclular $lim\int f_n$ para cada una de las siguientes sucesiones $\{ f_n \}$ de funciones
de $]0, 1[$ en $\R$:}

\[ f_n = \frac{nx}{1 + n^2x^2}         \hspace{1cm}
f_n = \frac{1}{n}log(x+n)e^{-x}cos(x)  \hspace{1cm}
f_n = \frac{1+nx}{(1+x)^n}
\]

{\ }

\textit{a)} $f_n = \frac{nx}{1 + n^2x^2} $
\[ (nx+1)^2 \geq 0 \implies n^2x^2 + 1 \geq 0 \implies nx/(1+n^2x^2) \leq \frac{1}{2}
\]
Hemos probado que existe una función, en concreto la constantemente igual a $1/2$ integrable en $]0, 1[$ tal que $|f_n| \leq 1/2$.
\[ \{ f_n(x) \} \rightarrow f=0 \hspace{0.5cm} \forall x \in ]0, 1[
\]
Por tanto estamos en condiciones de aplicar el teorema de la convergencia dominada
\[ lim\int_{]0, 1[} f_n = \int_{]0, 1[} f = \int_{[0, 1]} 0 = 0
\]

{\ }

\textit{b)} $f_n = \frac{1}{n}log(x+n)e^{-x}cos(x) $
\[ |f_n(x)| \leq \left| \frac{log(x+n)e^{-x}}{n} \right|
\]
Como $log(x) \leq x \ \forall x \in \R^+$
\[ \left| \frac{log(x+n)e^{-x}}{n} \right| \leq \frac{(x+n)e^{-x}}{n} =
	\left( \frac{x}{n} + 1 \right)e^{-x} \leq (x+1)e^{-x} \leq 1
\]
Sabemos que $\{f_n(x)\} \rightarrow 0 \ \forall x\in]0, 1[$, tenemos una sucesión de funciones 
integrables en $]0, 1[$ acotadas en valor absoluto por la función constantemente igual a $1$, integrable en $]0, 1[$, estamos en conficiones de aplicar el teoream de la convergencia dominada.
\[ lim \int_{]0, 1[} f_n = \int_{]0, 1[} 0 = 0
\]

% $0 \leq \left| \int_Ef - \int_Ef_n \right| \leq \left| \int_Ef-f_n \right| \leq \int_E \left| f-f_n \right|$ % Lo tenía apuntado, creo que no aporta nada.

{\ }

\textit{c)} $f_n = \frac{1+nx}{(1+x)^n}$

Como $1+nx \leq (1+x)^n$, entonces $|f_n(x)| \leq 1$, aplicamos de nuevo teorema de la convergencia dominada.
\[ \left\{ \frac{1+nx}{(1+x)^n} \right\} \rightarrow 0\ \implies \ lim\int_{]0, 1[} f_n = \int_{]0, 1[} 0 = 0
\]
	%\subsubsection{Ejercicio 5}
	%\enunciado{ Sea $E\subseteq \R^n$ un conjunto medible de medida finita y sea $\{ f_n \}$ una sucesión de funciones integrables en $E$ que converge uniformemente a una función $f$. Probar que $f$ es integrable y que
\[ lim \int_E|f-f_n| = 0
\]
Dar un ejemplo mostrando que la hipótesis de medida finita no puede ser suprimida.
} 

\[ f_n \mbox{ converge uniformemente a } f \mbox{ en E y } \lambda (E) < +\infty
\]
\[ \forall \epsilon > 0 \ \exists n_O > 0: n\geq n_0 \ |f(x)-f_n(x)| < \epsilon
\]
\[ f_n(x) - \epsilon \geq f(x) \geq f_n(x) + \epsilon \ \implies \ 
  | f_n(x | \leq | f_n(x) + \epsilon |
\]
Fijando $\epsilon = 1$
\[ \int_E |f(x)dx| \leq \int_E |f_n(x)+1|dx \leq 
   \int_E |f_n(x)|dx + \int_E1dx < +\infty
\]
vemos entonces que $f(x) \ \forall x\in E$ es integrable, y al haber fijado 
$\epsilon , |f(x)-f_n(x)| < 1$ y por el teorema de la convergencia dominada
\[ lim_{n \rightarrow \infty} \int_E |f-f_n| = 0
\]  

{\ }

La hipótesis de medida finita es necesaria:

Definamos en su caso el conjunto $E=\R$, con $f_n = \mathcal{X}_{]0,n]}$ 
% Tenía que era cerrado por la izquierda... (?)
$\{f_n\}$ es una sucesión de funciones integrables, ya que
\[ \int_{\R^+} \frac{1}{n} \mathcal{X}_{]0, n]} = 
   \frac{1}{n} \int_{\R^+} \mathcal{X}_{]0, n]} = \frac{n}{n} = 1 < +\infty
\] 
Sin embargo la función límite no es integrable
\[ f(x) = \left\{
		  \begin{matrix} 
    		      \frac{1}{x} & \mbox{ si }x\in ]0,n[
	          \\ 0        & \mbox{ si } x\not\in ]0, n[      	  
		  \end{matrix} 
		  \right. 
\]


	%\subsubsection{Ejercicio 6}
	%\enunciado{ Para cada natural $n$, y para cada $x\in\R$, sea $f_n = a_nsen(nx) + b_ncos(nx)$, donde $\{a_n\}$ y $\{b_n\}$ son dos sucesiones de númros reales. Pruébese que si la sucesión $\{f_n\}$ converge c.p.d. a uno en $[0, 2\pi]$ entonces la sucesión $\{|a_n| + |b_n|\}$ no está acotada.}

La hipótesis inicial es la convergencia c.p.d. a $1$ de $\{f_n\}$, supongamos que en ese caso $\{|a_n| + |b_n|\}$ está acotada:
\[\exists M>0, |f_n| \leq |a_n| + |b_n| \leq M \forall x\in [0, 2\pi] \]
$|f_n|$ está acotada por un número $M>0$, como la función constantemente igual a $M$ en $[0, 2\pi]$
es integrable, podemos aplicar el teorema de la convergencia dominada.
\[ lim_{n\to\infty}\int_{[0, 2\pi]} f_n(x)dx = \int_{[0, 2\pi]} lim_{n\to\infty} f_n(x)dx
\]
$ \int_{[0, 2\pi]} lim_{n\to\infty} f_n(x)dx = \int_{[0, 2\pi]} 1dx = 2\pi - 0 = 2\pi $

\begin{equation*}
\begin{split}
lim_{n\to\infty} \left( \int_[0, 2\pi] f_n(x)dx \right)
  & = lim_{n\to\infty} \left( \int_{0, 2\pi} a_nsen(nx)dx + \int_{0, 2\pi} b_ncos(nx)dx \right)
\\& = lim_{n\to\infty} \left( a_n\int_{0, 2\pi} sen(nx)dx + b_n\int_{0, 2\pi} cos(nx)dx \right)
\\& = lim_{n\to\infty} \left( -a_n \left[ \frac{cos(nx)}{n} \right]_0^{[2\pi]} 
					   + b_n \left[ \frac{sen(nx)}{n} \right]_0^{[2\pi]} \right)
\\& = lim_{n\to\infty} \frac{-a_n}{n} = 0
\end{split}
\end{equation*}
  
En resumen
\[ \int_{[0, 2\pi]} lim_{n\to\infty} f_n(x)dx = 2\pi \not =
	lim_{n\to\infty} \left( \int_[0, 2\pi] f_n(x)dx \right) = 0
\]
Por tanto la suposición de que $\{|a_n| + |b_n|\}$ estaba acotada es falsa.
  

	%\subsubsection{Ejercicio 7}
	%\enunciado{ Sea $E\subseteq \R^n$ un conjunto medible y $f$ una función integrable en $E$. 
Probar que 
\[ n\lambda \left( \{ x\in E: f(x) \geq n \}\right)\to 0
\]
Si $f$ no es integrable, ¿es cierto la anterior afirmación para funciones medibles positivas? En caso negativo, dése un contraejemplo.
} 
 
...








Si $f$ no es integrable cogemos $E = \R^+$ y $f = Identidad$.

Vemos que $\lambda  \{ x\in\R^+: f(x) \geq n\} )$ diverge.

	%\subsubsection{Ejercicio 8}
	%\enunciado{
Pruébese que la función $\frac{sen(x)}{x}$ es integrable en el intervalo $[0,1]$ y que
$$\int_{0}^{1}\frac{sen(x)}{x}dx = \sum_{n=0}^{\infty}\frac{(-1)^n}{(2n+1)(2n+1)!}$$}


Sabemos que el límite de la función en la abscisa $x=0$ es 1, por L'Hôpital. También podemos sacar que $\frac{sen(x)}{x}$ es una función decreciente en ese intervalo si estudiamos las derivadas, y además positiva, porque los dos términos lo son. Por lo tanto, sabemos que esta función esta acotada por 1 en el intervalo $[0,1]$. Siendo una función continua, concluimos que es integrable.

La idea de la igualdad va a ser aplicar el Teorema de la convergencia absoluta. Definiremos la sucesión de funciones $f_n$ como la expresión por el desarrollo de Taylor del $sen(x)$, transformándolo para obtener una expresión de $\frac{sen(x)}{x}$:
$$sen(x) =\sum_{n=0}^{\infty}\frac{(-1)^n}{(2n+1)!}x^{2n+1} \hspace{0.5cm} \forall x \in (0,1) \Rightarrow \frac{sen(x)}{x} =\sum_{n=0}^{\infty}\frac{(-1)^n}{(2n+1)!}x^{2n}\hspace{.5cm} \forall x \in (0,1) $$

La igualdad se da en el intervalo abierto, pero como se distingue del cerrado en un conjunto numerable de puntos, podemos estudiar la integral en este intervalo, que será la misma.
Definimos entonces $f_n : (0,1) \rightarrow \mathds{R}$:

$$f_n(x) = \frac{(-1)^n}{(2n+1)!}x^{2n}$$

Estas funciones son continuas, crecientes, conforman una serie de potencias y nos permiten bastante libertad para operar con ellas. Para usar el Teorema de la convergencia absoluta, vamos a probar que la suma del valor absoluto de las integrales de las $f_n$ en $(0,1)$ está acotada:

$$\sum_{n=0}^{\infty}\int_{0}^{1}|f_n(x)|dx = \sum_{n=0}^{\infty}\int_{0}^{1}\frac{1}{(2n+1)!}x^{2n}dx \leq \sum_{n=0}^{\infty}\int_{0}^{1}\frac{1}{(2n+1)!}dx = \sum_{n=0}^{\infty}\frac{1}{(2n+1)!} \leq \sum_{n=0}^{\infty}\frac{1}{(n+1)^2}$$

Y como esta última serie converge, sabemos que la primera también lo hace. Aplicando entonces el teorema, tenemos que:

$$\int_{0}^{1}\left(\sum_{n=0}^{\infty}f_n(x)\right)dx = \sum_{n=0}^{\infty}\left(\int_{0}^{1}f_n(x) dx\right)$$

Sustituimos $\frac{sen(x)}{x}$ con la sumatoria, usando la igualdad antes mencionada. Además, la integral de la serie de potencias de la derecha podemos hacerla por Riemann, o usando la regla de Barrow (que aparece en el siguiente tema), hallamos fácilmente la integral de cada $f_n$ en $(0,1)$, sabiendo que la función cuya derivada es $f_n$ es $\frac{(-1)^n}{(2n+1)(2n+1)!}x^{2n+1}$:
$$\int_{0}^{1}\frac{sen(x)}{x}dx = \sum_{n=0}^{\infty}\left(\int_{0}^{1}f_n(x) dx\right) =\sum_{n=0}^{\infty}\frac{(-1)^n}{(2n+1)(2n+1)!} $$

	%\subsubsection{Ejercicio 9}
	%\enunciado{Pruébese que la función $e^{x^2}$
 es integrable en el intervalo $[0,1]$ y que
\[ \int_0^1e^{x^2}dx = \sum_{n=0}^{\infty} \frac{1}{(2n+1)n!}
\]} 

\[ \int_0^1|e^{x^2}| \leq \int_0^1e = e < +\infty
\]
Por tanto $e^{x^2}$ es integrable en $[0, 1]$. Definimos $f(y)=e^y:\mathbb{R}\rightarrow\mathbb{R} \hspace{1cm} f\in C^{\infty}(\mathbb{R})$

Desarrollo en serie de Taylor con $a=0$
\[ e^y = \sum_{n=0}^{\infty} \frac{f^{n)}(a)}{n!}(y-a)^n=
\sum_{n=0}^{\infty} \frac{y^n}{n!}
\] y por tanto $e^{x^2}$ se puede expresar como $\sum_{n=0}^{\infty} \frac{x^{2n}}{n!}$

Comprobamos que podemos utilizar el teorema de la convergencia absoluta (*).
\[   \sum_{n=0}^{\infty} \int_0^1 \left| \frac{x^{2n}}{n!} \right| dx
 =   \sum_{n=0}^{\infty} \frac{1}{n!} \left[ \frac{x^{2n+1}}{2n+1} \right]_0^1
 =   \sum_{n=0}^{\infty} \frac{1}{(2n+1) n!} 
\leq \sum_{n=0}^{\infty} \frac{1}{2n^2} < +\infty
\]
Por tanto
\[ \int_0^1 e^{x^2} dx
 = \int_0^1 \sum_{n=0}^{\infty} \left| \frac{x^{2n}}{n!} \right| dx
 = \int_0^1 \sum_{n=0}^{\infty} \frac{x^{2n}}{n!} dx
 =^{(*)} \sum_{n=0}^{\infty} \int_0^1 \frac{x^{2n}}{n!} dx
 = \sum_{n=0}^{\infty} \frac{1}{(2n+1) n!}
\]
	%\subsubsection{Ejercicio 10}
	%\enunciado{Sean I,J dos intervalos abiertos de $\R$ y $f:IxJ \to \R$ una función continua verificando:
}

\textit{a)} Para cada $t\in I$, la función $x\to f(t,x)$ es integrable en $J$.

\textit{b)} Para cada $x\in J$, la función $t\to f(t,x)$ es de clase $\mathcal{C}^1$ en $I$.

\textit{c)} La función $x\to \left| \frac{\partial f(t,x)}{\partial t}\right|$ está dominada por una función integrable en $J$.

\textit{d)} Sea $g:I\to J$ una función de clase $\mathcal{C}^1$ en $I$.

Probar que para $a\in J$, la función 
\[ F(t) := \int_a^{g(t)} f(t,x)dx \]
es derivable en $I$ y que, para cada $t\in I$,
\[ F'(t) := \int_a^{g(x)} \frac{\partial f(t,x)}{\partial	 t} dx + f(t, g(t)) g'(t)\]


\newpage
\section{T\'ecnicas de integraci\'on}
\subsection{T\'ecnicas de integraci\'on en una variable}
	%\subsubsection{Ejercicio 1}
	%\input{./Ejercicios/Relacion3.1/Rel_3.1_ej1.tex}
	%\subsubsection{Ejercicio 2}
	%\input{./Ejercicios/Relacion3.1/Rel_3.1_ej2.tex}
	%\subsubsection{Ejercicio 3}
	%\input{./Ejercicios/Relacion3.1/Rel_3.1_ej3.tex}
	%\subsubsection{Ejercicio 4}
	%\input{./Ejercicios/Relacion3.1/Rel_3.1_ej4.tex}
	%\subsubsection{Ejercicio 5}
	%\enunciado{ Justificar, haciendo uso en cada caso de un conveniente teorema de convergencia, las siguientes igualdades:}

%%%%%%%%%%%%%%%%%%%%%%%%
\textit{d)} $\int_0^{+\infty} \frac{x}{1+e^x}dx = \sum_{n=1}^{\infty} \frac{(-1)^{n+1}}{n^2}$

Expresamos la función inicial de la siguiente forma
\[ \frac{x}{e^x} \left( \frac{1}{1+e^{-x}} \right)\]
Lo expresamos así para poner la segunda parte como una suma infinita de los términos de una progresión geométrica. Esta expresión es válida cuando 
$|-e^{-x}| < 1 \Leftrightarrow x \in ]0, +\infty$, como los puntos de $[0, +\infty]$ donde no se puede expresar la función de esa forma son numerables (solamente falla en el $0$) la integral será la misma.

\[ \frac{x}{e^x} \sum_{n=0}^{+\infty} (-1)^{n+1}e^{-nx} = x \sum_{n=1}^{+\infty} (-1)^{n+1}e^{-nx}
\]
Comprobamos que podemos usar el teorema de la convergencia absoluta (*).   
\[ \sum_{n=1}^{\infty} \int_0^{+\infty} \left| (-1)^{n+1}xe^{-nx} \right| dx
= \sum_{n=1}^{\infty} \int_0^{+\infty} xe^{-nx} dx
\]

Integramos por partes 
\[ \int_0^{+\infty} xe^{-nx}dx \hspace{1cm}
	\begin{bmatrix}
	f  = x    &  dg = e^{-nx}dx \\
	df = dx   &   g = \frac{e^{-nx}}{-n}
	\end{bmatrix}
\]
Cambio de variable
\[ -\frac{xe^{-nx}}{n} + \frac{1}{n} \int_0^{+\infty}e^{-nx}dx \hspace{1cm} 
	\begin{bmatrix}
	u = -nx   \\
	du = -ndx 
	\end{bmatrix}
\]
\[ -\frac{xe^{-nx}}{n} + \frac{1}{n^2} \int_0^{+\infty}e^udx 
=  \left[ -\frac{e^{-nx}(nx+1)}{n^2} \right]_0^{+\infty} 
= \frac{1}{n^2}
\]
Como $\sum_{n=0}^{+\infty} \frac{1}{n^2}$ converge podemos intercambiar el sumatorio con la integral.

\[ \int_0^{+\infty} \frac{x}{1+e^x}dx 
 = \int_0^{+\infty} \sum_{n=1}^{+\infty} (-1)^{n+1}xe^{-nx}   dx
 =^{(*)} \sum_{n=1}^{+\infty} (-1)^{n+1}  \int_0^{+\infty} xe^{-nx} dx
 = \sum_{n=1}^{+\infty} \frac{(-1)^{n+1}}{n^2}
\]




%%%%%%%%%%%%%%%%%%%%

\textit{e)} $\int_0^{+\infty} \frac{e^{-ax}}{1+e^{-bx}}dx = \sum_{n=0}^{+\infty} \frac{(-1)^n}{a+nb}$
$(a,b>0)$ y deducir que 
\[ \sum_{n=0}^{+\infty} \frac{(-1)^n}{2n+1} = \frac{\pi}{4} \hspace{0.5cm}
   \sum_{n=0}^{+\infty} \frac{(-1)^n}{n+1}  = ln(2) 
\]

Empecemos expresando el integrando como una constante y la suma infinita de una progresión geométrica.
Como el en caso anterior esta expresión es válida $\forall x\in ]0, +\infty[$, al diferenciarse con
la original en un número finito de puntos las integrales son iguales.
\[ \frac{e^{-ax}}{1+e^{-bx}}
 = e^{-ax}\frac{1}{1+e^{-bx}}
 = e^{-ax} -\sum_{n=0}^{+\infty} (-e^{-bx})^n
 = -\sum_{n=0}^{+\infty} (-1)^n e^{-ax-bnx}
\]

¿Podemos usar el teorema de la convergencia absoluta?
\[ \sum_{n=0}^{+\infty} \int_0^{+\infty} | (-1)^ne^{-ax-bnx} |
 = \sum_{n=0}^{+\infty} \int_0^{+\infty} e^{-ax-bnx}
 = -\sum_{n=0}^{+\infty} \left[ \frac{e^{-ax-bnx}}{a+bn} \right]_0^{+\infty} 
 = -\sum_{n=0}^{+\infty} \frac{1}{a+bn}
\]

Como el sumatorio anterior diverge no podemos usar el teorema de la convergencia absoluta, probemos
con el teorema de la convergencia dominada.

Definimos nuestra sucesión de funciones integrables $f_n(x) = \sum_{i=0}^{n} \frac{(-1)^i}{e^{ax+bix}}$.
Es claro que
\[ \{f_n(x)\} \to \sum_{i=0}^{+\infty} \frac{(-1)^i}{e^{ax+bix}}
\]



Vamos a intentar acotar los elementos de la sucesión en valor absoluto
\[ |f_n(s)| \leq \sum_{i=0}^{E(n)} \frac{(-1)^i}{e^{ax+bix}} \hspace{1cm} \forall n,x\in\R
\]







\textbf{Deducciones}

- Caso $a=1m¡,b=2$
\[ \sum_{n=0}^{+\infty} \frac{(-1)^n}{1+2n}
 = \int_0^{+\infty} \frac{e^{-x}}{1+e^{-2x}}dx \hspace{1cm}
	\begin{bmatrix}
	u  = e^x  \\
	du = e^x dx
	\end{bmatrix}
\]
\[ = \int_0^{+\infty} \frac{1}{\left( \frac{1}{u^2} +1 \right) u^2}du 
   = \int_0^{+\infty} \frac{1}{1+s^2} ds
    \hspace{1cm}
	\begin{bmatrix}
	s  = \frac{1}{u} \\
	ds = -\frac{1}{u^2}du	
	\end{bmatrix}
\]
Integramos y deshacemos los cambios de variable
\[ -tan^{-1}(s) \hspace{0.5cm} -tan^{-1}\left( \frac{1}{u}\right) \hspace{0.5cm}
   -tan^{-1} (e^{-x})
\]
Por tanto
\[ \int_0^{+\infty} \frac{e^{-x}}{1+e^{-2x}}dx
 = \left[ -tan^{-1} (e^{-x}) \right]
 = -(0-\frac{\pi}{4}) = 	\frac{\pi}{4}
\]


- Caso $a=b=1$
\[ \sum_{n=0}^{+\infty} \frac{(-1)^n}{n+1} 
 = \int_0^{+\infty} \frac{e^{-x}}{1+e^{-x}} dx \hspace{1cm}
	\begin{bmatrix}
	u  = -x \\
	du = -dx
	\end{bmatrix}
\]
\[ = - \int_0^{+\infty} \frac{e^u}{1+e^u} du 
   = - \int_0^{+\infty} \frac{ds}{s} \hspace{1cm}
	\begin{bmatrix}
	s = e^u + 1 \\
	ds = e^u du
	\end{bmatrix}
\]
Hacemos los correspondientes cambios de variable
\[ -ln(s) \hspace{0.5cm} -ln(e^u+1) \hspace{0.5cm} -ln(e^{-x}+1)
\]
\[ \int_0^{+\infty} \frac{e^{-x}}{1+e^{-x}} dx 
 = \left[ -ln(e^{-x}+1) \right]_0^{+\infty} = -ln(2)
\]

	%\subsubsection{Ejercicio 6}
	%\input{./Ejercicios/Relacion3.1/Rel_3.1_ej6.tex}
	%\subsubsection{Ejercicio 7}
	%\input{./Ejercicios/Relacion3.1/Rel_3.1_ej7.tex}
	
\subsection{T\'ecnicas de integraci\'on en varias variables}
	%\subsubsection{Ejercicio 1}
	%\input{./Ejercicios/Relacion3.2/Rel_3.2_ej1.tex}
	%\subsubsection{Ejercicio 2}
	%\input{./Ejercicios/Relacion3.2/Rel_3.2_ej2.tex}
	%\subsubsection{Ejercicio 3}
	%\input{./Ejercicios/Relacion3.2/Rel_3.2_ej3.tex}
	%\subsubsection{Ejercicio 4}
	%\input{./Ejercicios/Relacion3.2/Rel_3.2_ej4.tex}
	%\subsubsection{Ejercicio 5}
	%\input{./Ejercicios/Relacion3.2/Rel_3.2_ej5.tex}
	%\subsubsection{Ejercicio 6}
	%\input{./Ejercicios/Relacion3.2/Rel_3.2_ej6.tex}
	%\subsubsection{Ejercicio 7}
	%\input{./Ejercicios/Relacion3.2/Rel_3.2_ej7.tex}
	%\subsubsection{Ejercicio 8}
	%\input{./Ejercicios/Relacion3.2/Rel_3.2_ej8.tex}
	%\subsubsection{Ejercicio 9}
	%\input{./Ejercicios/Relacion3.2/Rel_3.2_ej9.tex}
	%\subsubsection{Ejercicio 10}
	%\input{./Ejercicios/Relacion3.2/Rel_3.2_ej10.tex}
	%\subsubsection{Ejercicio 11}
	%\input{./Ejercicios/Relacion3.2/Rel_3.2_ej11.tex}		
	%\subsubsection{Ejercicio 12}
	%\input{./Ejercicios/Relacion3.2/Rel_3.2_ej10.tex}
	%\subsubsection{Ejercicio 13}
	%\input{./Ejercicios/Relacion3.2/Rel_3.2_ej11.tex}

%%%%%%%%%%%%%%%%%%%%%%%%%%%%%%%%%%%%%%%%%%%%%%%%%%%%%%%%%%%%%%%%%%%%%%%%%%%%%%%%%%

% % % % % % % % % % % % % % % % % % % % % % % % % % % % % % % % %
%					 Bibliografía
% % % % % % % % % % % % % % % % % % % % % % % % % % % % % % % % %

\end{document}
