\enunciado{Sean $(\Omega, \mathcal{A})$ un espacio medible y sea $E\in\mathcal{A}$.} 

\textit{a)} 
Si $f:\Omega \rightarrow \mathbb{R}$ es una función medible entonces $f|_E$ es una función medible cuando se considera el espacio medible $(E,\mathcal{A}_E)$.

$$\mathcal{A}_E = \{ A\cap E: A\in\mathcal{A} \}$$
$f$ medible $\implies f^{-1}(B)\in\mathcal{A} \hspace{1cm} \forall B \in \mathbb{B}$ 

$E\subseteq\Omega \implies f^{-1}_E(B)= f^{-1}(B)\cap E \in \mathcal{A}_E \hspace{1cm} \forall B\in\mathcal{B}$
$\implies f|_E$ es medible en $(E, \mathcal{A}_E)$

\ 

\textit{b)} Sea $f:E\rightarrow\mathbb{R}$ una función. Probar que $f$ es medible si, y sólo si, la función $f_{\mathcal{X} E}:\Omega\rightarrow\mathbb{R}$, definida por
\[ f_{\mathcal{X} E}(x) = \left\{
	\begin{matrix}
		f(x)			& \mbox{si } x\in E
		\\0			& \mbox{si } x\not\in E
	\end{matrix} \right.
\]
es medible. Además si $\Omega = \mathbb{R}^N$ entonces
\[ \int_Efd\lambda = \int_{\mathbb{R}^N} f \mathcal{X}E d\lambda\]

$\Longleftarrow (f_{\mathcal{X}E} \ medible \implies f \ medible)$

Por el apartado anterior $f_{\mathcal{X}E}|_E = f$ es medible.

$\vspace{0.5cm}$
$\Longrightarrow (f\ medible \implies f_{\mathcal{X}E} \ medible)$

\[ f^{-1}_{\mathcal{X}E}(B) = \left\{
	\begin{matrix}
		f^{-1}(B)\cap E & \mbox{si } 0\not\in B
		\\ E^c\cup ( f^{-1}(B)\cap E ) & \mbox{si } 0\in B 
	\end{matrix} \right.	
\] 
Por las propiedades de las $\sigma$-álgebra y los espacios medibles:

$f^{-1}(B)\cap E \in \mathcal{A}$ y $E^c\cup ( f^{-1}(B)\cap E )\in \mathcal{A} \implies f^{-1}_{\mathcal{X}E}(B) \in \mathcal{A}$

\

Como $\Omega = \mathbb{R}^N$

$\int_{\mathbb{R}^N}f_{\mathcal{X}E}d\lambda  = \int_Efd\lambda + \int_{E^c}0d\lambda$
$= \int_Efd\lambda + \lambda(R(0)) = \int_Efd\lambda$

\

\textit{c)} Sea $f:E\rightarrow\mathbb{R}$ una función medible. Probar que $f$ es integrable en $E$ si, y sólo si, $f_{\mathcal{X}E}$ es integrable en $\mathbb{R}^N$

Probando que $|f|_{\mathcal{X}E} = |f_{\mathcal{X}E}|$ por el apartado anterior sabemos que $\int|f|d\lambda = \int|f_{\mathcal{X}E}|d\lambda	$

\[ |f|_{\mathcal{X}E}(x) = \left\{
	\begin{matrix}
		|f(x)|	 & \mbox{si } x\in E
		\\	0	 & \mbox{si } x\not\in E	
	\end{matrix} \right.
	= \left\{
	\begin{matrix}
		f(x) 	& \mbox{si } f(x)>0
		\\-f(x) & \mbox{si } f(x)<0
		\\ 0		& \mbox{si } x\not\in E \mbox{ ó } f(x)=0
	\end{matrix} \right.
\]

\[ |f_{\mathcal{X}E}(x)| = \left\{
	\begin{matrix}
		f_{\mathcal{X}E}(x)  & \mbox{si } f_{\mathcal{X}E}>0
		\\ -f_{\mathcal{X}E}(x)  & \mbox{si } f_{\mathcal{X}E}<0
		\\ 0  & \mbox{si } f_{\mathcal{X}E}=0
	\end{matrix} \right.
	= \left\{
	\begin{matrix}
		f(x) 	& \mbox{si } f(x)>0
		\\-f(x) & \mbox{si } f(x)<0
		\\ 0		& \mbox{si } x\not\in E \mbox{ ó } f(x)=0
	\end{matrix} \right.
\]