\enunciado{Pruébese que la función $e^{x^2}$
 es integrable en el intervalo $[0,1]$ y que
\[ \int_0^1e^{x^2}dx = \sum_{n=0}^{\infty} \frac{1}{(2n+1)n!}
\]} 

\[ \int_0^1|e^{x^2}| \leq \int_0^1e = e < +\infty
\]
Por tanto $e^{x^2}$ es integrable en $[0, 1]$. Definimos $f(y)=e^y:\mathbb{R}\rightarrow\mathbb{R} \hspace{1cm} f\in C^{\infty}(\mathbb{R})$

Desarrollo en serie de Taylor con $a=0$
\[ e^y = \sum_{n=0}^{\infty} \frac{f^{n)}(a)}{n!}(y-a)^n=
\sum_{n=0}^{\infty} \frac{y^n}{n!}
\] y por tanto $e^{x^2}$ se puede expresar como $\sum_{n=0}^{\infty} \frac{x^{2n}}{n!}$

Comprobamos que podemos utilizar el teorema de la convergencia absoluta (*).
\[   \sum_{n=0}^{\infty} \int_0^1 \left| \frac{x^{2n}}{n!} \right| dx
 =   \sum_{n=0}^{\infty} \frac{1}{n!} \left[ \frac{x^{2n+1}}{2n+1} \right]_0^1
 =   \sum_{n=0}^{\infty} \frac{1}{(2n+1) n!} 
\leq \sum_{n=0}^{\infty} \frac{1}{2n^2} < +\infty
\]
Por tanto
\[ \int_0^1 e^{x^2} dx
 = \int_0^1 \sum_{n=0}^{\infty} \left| \frac{x^{2n}}{n!} \right| dx
 = \int_0^1 \sum_{n=0}^{\infty} \frac{x^{2n}}{n!} dx
 =^{(*)} \sum_{n=0}^{\infty} \int_0^1 \frac{x^{2n}}{n!} dx
 = \sum_{n=0}^{\infty} \frac{1}{(2n+1) n!}
\]