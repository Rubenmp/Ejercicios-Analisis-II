\enunciado{Sea $(\Omega, \mathcal{A})$ un espacio de medida, $\{f_n\}$ una sucesión de funciones medibles y $f,g$ dos funciones medibles. Porbar las siguientes afiirmaciones:}


\textit{a)}
	Si $\{f_n\}$ converge c.p.d. a $f$ y a $g$ c.p.d. entonces $f=g$ c.p.d. 

Definimos
\[A = \{ x\in\Omega / \{f_n(x)\} \rightarrow f(x) \} \hspace{0.5cm}
  B = \{ x\in\Omega / \{f_n(x)\} \rightarrow g(x) \} \hspace{0.5cm}
\]

\[  D = \{ x\in\Omega / f(x) = g(x) \} 
\]
Sabemos por hipótesis que $\mu (A^c) = \mu (B^c) = 0$ y si $x\in (A\cap B) \implies f(x) = g(x)$.

\[D^c \subseteq (A\cap B)^c = A^c \cup B^c \]
Suponemos que $D^c$ es medible, por tanto
$\mu (D^c) \leq \mu (A^c \cup B^c) \leq \mu(A^c) + \mu (B^c) = 0$

Por tanto $\mu (D^c)=0$ y $f=g$ c.p.d. 

{\ } 
 
\textit{b)}
	Si $\{f_n\}$ converge c.p.d. a $f$ y $f=g$ c.p.d. entonces $\{f_n\}$ converge c.p.d. a $g$.
	
Con los conjuntos anteriores sabemos que $\mu (A^c) = 0$, $\mu (D^c) = 0$ y en este caso
\[Si \ x\in D, entonces \ x\in B \implies \mu (B^c) \leq \mu (D^c) = 0\]
Por tanto $\{f_n\}$ converge c.p.d. a $g$.
