\enunciado{
Sea, para cada $n\in\mathds{N}$,
$$f_n(x) = \frac{x}{n^\alpha(1+nx^2)} \hspace{0.5cm} (x\geq0)$$
Prueba que la serie $\sum f_n$ converge}

\textit{a)} puntualmente en $\mathds{R}^+_0$ si $\alpha > 0$
\textit{b)} uniformemente en semirrectas cerradas que no contienen al cero
\textit{c)} uniformemente en $\mathds{R}^+_0$ si $\alpha > 1/2$

\textit{a)} 
Podemos escribir las funciones $f_n$ como:

$$f_n(x) = \frac{x}{n^\alpha+n^{\alpha+1}x^2} $$

Ahora, fijado un $x\in\mathds{R^+}$, usamos el criterio de comparación por paso al límite con la serie $\frac{1}{n^{\alpha+1}}$, que sabemos que converge a un número real, puesto que el exponente de la $n$ en el denominador es mayor estricto que 1:

$$\lim\limits_{n->\infty} \frac{\frac{x}{n^\alpha+n^{\alpha+1}x^2}}{\frac{1}{n^{\alpha+1}}} = \lim\limits_{n->\infty} \frac{x}{\frac{1}{n}+x^2} = \frac{1}{x} $$

Como $1/x$ es un número real distinto de 0, por el criterio sabemos que nuestra serie con los términos $f_n(x)$ también converge (le pasa lo mismo que con la que la hemos comparado).

Para el caso en que $x$ sea 0, es fácil ver que la serie converge puntualmente, puesto que vale 0.


\textit{b)}  Sea la semirrecta $[c, \infty)$, con $c > 0$ podemos ver que:

$$f_n(x) = \frac{x}{n^\alpha+n^{\alpha+1}x^2} \leq \frac{x}{n^{\alpha+1}x^2} = \frac{1}{n^{\alpha+1}x} \leq \frac{1}{n^{\alpha+1}c}$$
Esta última serie es independiente de las $x\in [c, \infty)$ y converge (se puede comprobar por comparación paso al límite con $\frac{1}{n^{\alpha+1}}$ el límite es $c$, que es una constante mayor que 0). Por el test de Weierstrass, podemos afirmar que $f_n$ converge uniformemente en toda la semirrecta $[c, \infty)$.


\textit{c)} Hacemos las derivadas de $f_n$:

$$f_n'(x) = \frac{n^\alpha + n^{\alpha+1}x^2-2x^2n^{\alpha+1}}{(n^\alpha+n^{\alpha+1}x^2)^2} = \frac{n^\alpha(1 - nx^2)}{(n^\alpha+n^{\alpha+1}x^2)^2}$$

Las funciones $f_n$ son crecientes hasta $x=\frac{1}{\sqrt{n}}$, donde empiezan a decrecer, así que esta abscisa es un máximo de las $f_n$. De nuevo, podemos acotar la serie por una sucesión de números reales $a_n$ y concluir que covergen uniformemente en $\mathds{R}^+_0$. Definimos la sucesión de los máximos de la función: $a_n = f_n\left(\frac{1}{\sqrt{n}}\right)$. Vemos que:

$$f_n(x) \leq a_n = f_n\left(\frac{1}{\sqrt{n}}\right) = \frac{\frac{1}{\sqrt{n}}}{n^\alpha(1+1)} = \frac{1}{2n^{\alpha+1/2}} \hspace{0.5cm} \forall x \in \mathds{R}^+_0$$
La serie de términos $a_n$ converge ya que $(\alpha + 1/2) > 1$, así que por el test de Weierstrass, la serie $f_n$ converge uniformemente en $\mathds{R}^+_0$.
	