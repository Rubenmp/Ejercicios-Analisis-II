\enunciado{Estudio la convergencia uniforme en intervalos de la forma $[0,a]$ y $[a,+\infty[$, donde $a>0$, de la sucesión de funciones $\{ f_n \}$ definidas para todo $x \geq 0$ por:}

\[f_n(x) = \frac{2nx^2}{1+n^2x^4}\]

$$f_n(x) = \frac{2nx^2}{1+n^2x^4} \hspace{2cm} f_n'(x) = \frac{4nx(1+n^2x^4)-8n^3x^5}{(1+n^2x^4)^2}$$

Tenemos que $f_n'(x) = 0 \iff 4nx+4n^3x^5-8n^3x^5 = 0 \iff x-n^2x^5 = 0 \iff x=\left\{
\begin{array}{l}
0, \\
\frac{1}{\sqrt n}
\end{array}\right.$

Comprobamos el signo de la derivada en medio de los intervalos con algún punto intermedio, sustituyendo:
$$f_n'\left(\frac{\sqrt 2}{\sqrt n}\right) < 0 \hspace{2cm} f_n'\left(\frac{1}{\sqrt{2n}}\right) > 0$$

Es decir, que la sucesión de máximos la vamos a poder coger como $x_n = f_n\left(\frac{1}{\sqrt n}\right)$.

Este máximo se encuentra solo en el primer conjunto que queremos estudiar, $[0,a]$, donde utilizando lo que hemos visto arriba, si estudiamos esta sucesión terminamos:

$$x_n = \frac{2}{2} = 1$$

Como no converge a 0, no hay convergencia uniforme en $[0,a]$. Por otro lado, en el conjunto $[a, +\inf)$ las $f_n$ son decrecientes, así que esta vez tomaremos la sucesión de máximos como $x_n = f_n(a)$. Pero aquí lo tenemos más fácil. En caso de haber estudiado que existe convergencia puntual en todo $\mathds{R^+_0}$, en concreto habríamos probado que hay convergencia puntual en $a$, por lo que $\{x_n\}$ converge a 0.
